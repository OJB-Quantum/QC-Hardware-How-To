\large{This document is meant to provide some level of consolidation for those desiring to be involved with quantum hardware engineering. By doing one's best to maintain familiarity with these topics, it is possible to become one who designs, builds, tests, operates, and maintains real quantum machines - a quantum mechanic. Another possibility is to begin working on a doctorate degree in the associated field with these training resources on hand. There are many clickable links in this document, so it might be best to view it using a browser or PDF viewer. 
\\ 
\space
\\
\indent My decision to share these resources is because they have been useful to me in my PhD work. This has been a very interesting path for me as an tribesman from the Navaho Nation. Here is the path: carpenter $\Longrightarrow$ electric vehicle researcher $\Longrightarrow$ nanotechnologist $\Longrightarrow$ quantum mechanic.
\\ 
\space
\\
\indent Please note that open access is a key theme held herein. Enjoy. -Onri
}
\\
\begin{center}

\includegraphics[scale=0.75]{qrcode_www.overleaf.com.png}

Scan QR code to access digital downloadable version.

\end{center}

\begin{center}
\space
\end{center}


\begin{flushleft}
\title{\Large\textbf{Creative Commons License}}\\
\end{flushleft}

{
\large This work is licensed under the Creative Commons Attribution 4.0 International License. To view a copy of this license, visit \url{http://creativecommons.org/licenses/by/4.0/} or send a letter to Creative Commons, PO Box 1866, Mountain View, CA 94042, USA.
}

\begin{center}

\includegraphics{by.png}

\end{center}